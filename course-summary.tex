
\documentclass{article}

\usepackage{fullpage}
\usepackage{color}
\usepackage{url}

\title{\textbf{Model Checking and System Verification}}
\author{Instructor: Mandayam Srivas \\
TA: Zubin Duggal}
\date{Aug-Nov 2020}

\pagenumbering{gobble}

\begin{document}

\maketitle
\small{
\noindent
Techniques for ensuring that a reactive computing (hardware + software) system is free of bugs are of utmost importance since vulnerabilities in such systems when embedded in devices can often cause huge damage in real life.
Hence there is tremendous interest within computer science community and industry to find methods that can guarantee absence of bugs in systems when compared to traditional testing and simulation, which are able to only show presence of bugs.
Model Checking is one such promising automatic method which applies foundational concepts and techniques from symbolic logic and automata theory to formally verify a model of a system satisfies a desired property for all possible behaviors of the system.
Since its path-breaking Turing-award cited invention in 1986, several technical break-throughs have made model checking a viable technology for design validation in industrial practice while enhancements to it is continuing to be pursued as an active research area.
This course covers both the theory and techniques of model checking at an advanced undergraduate and beginning graduate level.
The course gives special emphasis on symbolic model checking with SAT solvers which is a most widely used version in practice and also forms the current research direction in this field.
The course will also give hands-on exposure to a few state-of-art model checking tools.
The course will give a quick recap Logic to cover the required background for the course. \\[0.25cm]
\noindent
\textbf{Learning Outcomes:}
\begin{itemize}
\item The basic theory of model checking in propositional temporal logics
\item Symbolic Model checking using SAT solvers
\item Abstraction and Inductive Techniques to scale model checking
\item Practical exercises on symbolic model checking tools
\end{itemize}
\noindent
\textbf{Targeted Audience:}
\begin{itemize}
\item People that want to pursue formal verification as an industrial career
\item People that are interested in learning about an exciting applied field of logic and automata
\item People that want to pursue a research career in formal verification and validation
\end{itemize}
\noindent
\textbf{Desired (not Required) Background:} Logic, Theory of computation, Algorithms \\[0.25cm]
\noindent
\textbf{Course Outline:}
\begin{enumerate}
\item Modeling Systems (Kripke Structures)
\item Specifying Properties (Temporal logics: CTL, LTL, CTL*)
\item Verifying Properties (Model Checking for Temporal logics)
\item Abstraction Techniques
\item Automata-Based Model Checking
\end{enumerate}
\noindent
\textbf{Course Work and Grade Distribution}
\begin{enumerate}
\item  Written Assignments: 30\%
%\begin{enumerate}
%\item Modeling and Temporal Logic
%\item SAT and BDD Exerzises
%\end{enumerate}

\item  Tool Exercises: 30\%
%\begin{enumerate}
%\item Verification exercizes on tools NuSMV, CBMC, SAT-ABS
%\item 
%\end{enumerate}

\item  2 Quizzes: 40\%
%\begin{enumerate}
%\item Quiz-1
%\item Quiz-2
%\end{enumerate}

\end{enumerate}
}

%\bibliographystyle{plain}
%\bibliography{refs}


\end{document}

